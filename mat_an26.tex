\documentclass[10pt]{article}
\pdfoutput=1
\usepackage{Listo4ki}


\begin{document}
	
	\setcounter{footnote}{0}
	\setcounter{section}{0}
	
	\setcounter{part}{26}
	%\maketitle
	\flushbottom
	\newpage
	\pagestyle{fancynotes}
	\part{Возрастание и убывание функции. Направление выпуклости.}
	\begin{margintable}\vspace{.8in}\footnotesize
		\begin{tabularx}{\marginparwidth}{|X}
			Секция~\ref{sec:theory_26}. Теория\\
			Секция~\ref{sec:problems_26}. Задачи\\
			%Section~\ref{sec:license}. Margins\\
		\end{tabularx}
	\end{margintable}

\section[Теория]{Теория}\label{sec:theory_26}

\vspace{1ex}

\begin{definition}
Функция $f$ \emph{возрастает} в точке $x=x_0$, если
$\exists \delta(x_0) > 0$, что выполнено $f(x) < f(x_0)$ при $x_0 - \delta < x < x_0$, и $f(x) > f(x_0)$ при $x_0 < x < x_0 + \delta$.
\end{definition}

\begin{definition}
 Функция $f$ \emph{убывает} в точке $x = x_o$, если
$\exists \delta(x_0) > 0$, что выполнено $f(x)>f(x_0)$ при $x_0 - \delta < x < x_0$, и $f(x) < f(x_0)$ при $x_0 < x < x_0 + \delta$.
\end{definition}

\begin{proposition}
 Пусть функция $f$ дифференцируема на интервале $(a; b)$.
Между характером монотонности данной функции и знаком её производной на этом интервале имеется следующая взаимосвязь:
$$f^{\prime}(x) > 0 \, \emph{(\text{дост})} \; \Rightarrow \, f(x)\!  \uparrow \; \emph{(\text{необх})} \; \Rightarrow \,  f^{\prime}(x) \geqslant 0;$$
 \vspace{-5ex}
$$\quad f^{\prime}(x) \geqslant 0 \, \Rightarrow \, f(x)\!\!\nearrow \; \Rightarrow \,  f'(x) \geqslant 0;$$
 \vspace{-4ex}
$$f^{\prime}(x) = 0 \, \Rightarrow \, f(x) = \textbf{const} \, \Rightarrow \, f^{\prime}(x) = 0;$$
 \vspace{-4ex}
$$f^{\prime}(x) < 0 \, \Rightarrow \, f(x)\! \downarrow \; \Rightarrow \, f^{\prime}(x) \leqslant 0;$$
 \vspace{-4ex}
$$ \quad f^{\prime}(x) \leqslant 0 \, \Rightarrow \, f(x)\!\! \searrow \; \Rightarrow \, f^{\prime}(x) \leqslant 0;$$
\end{proposition}

\begin{remark}
 Отметим, что строгая положительность производной не является необходимым условием возрастания функции.
Например, функция $f(x) = x^{3}$ в точке $x=0$ возрастает, но $f^{\prime}(x) = 0$.
\end{remark}


\begin{definition}
 Функция $f$ \emph{выпукла вверх на множестве $\textbf{X}$}, если 
 \vspace{-2ex}
$$\forall x_1, x_2 \in \{x\}; \forall \alpha_1, \alpha_2 \geqslant 0 \, : \, \alpha_1~+~\alpha_2~=~1 \, \Rightarrow f(\alpha_1 x_1 + \alpha_2 x_2) \geqslant \alpha_1 f(x_1) + \alpha_2 f(x_2).$$
\end{definition}

\begin{definition}
Функция $f$ \emph{выпукла вниз на множестве $\textbf{X}$}, если 
 \vspace{-2ex}
$$\forall x_1, x_2 \in \textbf{X}; \forall \alpha_1, \alpha_2 \geqslant 0 \, : \, \alpha_1~+~\alpha_2~=~1 \Rightarrow f(\alpha_1 x_1+\alpha_2 x_2) \leqslant \alpha_1 f(x_1)+\alpha_2 f(x_2).$$
\vspace{-3ex}
\end{definition}

%\hspace{-3.5ex}\underline{\textbf{Теорема (достаточное условие выпуклости):}}
%\begin{itemize}
%\item[\textbf{-}] Пусть функция $f(x)$ дважды дифференцируема на множестве $\{x\}$, и $f^{\prime \prime}(x) \le 0 \,$ для $\forall x~\in~\{x\}$, тогда $f(x)$ - \emph{выпукла вверх} на множестве $\{x\}.$

%\item[\textbf{-}] Пусть функция $f(x)$ дважды дифференцируема на множестве $\{x\}$, и $f^{\prime \prime}(x) \ge 0 \,$ для $\forall x~\in~\{x\}$, тогда $f(x)$ - \emph{выпукла вниз} на множестве $\{x\}.$
%\end{itemize}

\begin{proposition}[критерий выпуклости]

Пусть функция $f$ дважды дифференцируема на множестве $\textbf{X}$. Тогда, для того, чтобы данная функция была выпуклой вверх (выпуклой вниз), необходимо и достаточно, чтобы $f^{\prime \prime}(x) \le 0 \,$ $\bigl(f^{\prime \prime}(x) \ge 0 \,\bigr)$ для всех $x$ из множества $\textbf{X}$. Если же $f^{\prime \prime}(x) < 0 \,$ $\bigl(f^{\prime \prime}(x) > 0 \,\bigr),$ то этого достаточно, чтобы гарантировать \textsf{\textsl{строгую}} выпуклость вверх (вниз) данной функции на этом множестве.
\index{Критерий!выпуклости}
\end{proposition}

%\begin{itemize}
%\item[\textbf{-}] , и $f^{\prime \prime}(x) \le 0 \,$ для $\forall x~\in~\{x\}$, тогда $f(x)$ - \emph{выпукла вверх} на множестве $\{x\}.$

%\item[\textbf{-}] Пусть функция $f(x)$ дважды дифференцируема на множестве $\{x\}$, и $f^{\prime \prime}(x) \ge 0 \,$ для $\forall x~\in~\{x\}$, тогда $f(x)$ - \emph{выпукла вниз} на множестве $\{x\}.$
%\end{itemize}


\begin{definition}
 Точка $x$ является \emph{точкой перегиба функции} \emph{$f$}, если в данной точке $f$ меняет направление выпуклости.
\index{Точка перегиба}
\end{definition}

\begin{proposition}
Пусть функция $f$ дважды дифференцируема на множестве
$\{0<|x-x_0|<\delta\},\,$ $f^{\prime \prime}(x_0)~=~0$ или $\nexists f^{\prime \prime}(x_0),$ и, кроме того,
$f^{\prime \prime}(x_1) \cdot f^{\prime \prime}(x_2)<0$, где $x_0 - \delta < x_1 < x_0 < x_2 < x_0 + \delta.$ Тогда $x_0 ~-~\text{точка перегиба}$.
\end{proposition}

\section[Задачи]{Задачи}\label{sec:problems_26}

%\task Пусть $p \ge 1$. Докажите, используя неравенство Йенсена, что для любых неотрицательных $a_1, \ldots, a_n$ справедливо неравенство
%$$a_1^p \, + \, a_2^p \, + \dots + a_n^p \, \le \, (a_1 \, + \, a_2 + \dots + a_n)^p.$$

%\vspace{2ex}

%\hspace{-3.5ex}\underline{\textbf{Замечание:}} Сравните данное решение с тем, что было получено в №1.17.

\begin{problem}
  Определите интервалы монотонности \emph{(возрастания или убывания)} следующих функций:
\vspace{0ex}

\emph{(а)} $y=x^n \, e^{-x}, \quad (n > 0, \; x > 0);$ \hspace{6ex}
\emph{(б)} $y = \begin{cases}
x \Bigl(\sqrt{\frac{3}{2}}+\sin{(\ln{x}})\Bigr), &\text{при $x > 0$;}\\
\;\;\;\;\;\;\;\;\;\,0,&\text{при $x = 0$.}
\end{cases}
$
\vspace{1ex}

\emph{(в)}  $y = \cos{\dfrac{\pi}{x}};$ \hspace{23ex}
\emph{(г)}  $y = \left(1+\dfrac{1}{x}\right)^x, \quad x \in \mathbb{R} \backslash [-1;0];$
\vspace{3ex}

\emph{(д)} $y = \sqrt{x-2} \, + \, \sqrt{x-1} \, + \, \sqrt{x+2} \, + \, \sqrt{x+1} \, - \, 2\sqrt{x}, \qquad x \geqslant 2.$
\end{problem}

%\begin{center}
%\hspace{0ex} \includegraphics[width=1.05\textwidth]{ris_26b.jpg}
%\end{center}

\begin{problem}
Пусть $f: \mathbb{R}\rightarrow \mathbb{R}, \,\, g: \mathbb{R}\rightarrow \mathbb{R}$ - произвольные монотонно возрастающие на множестве $\mathbb{R}$ функции. Какие из функций вида $f + g, \, f - g, \, f \cdot g, \, f(g)$  необходимо монотонно возрастают на $\mathbb{R}$?
\end{problem}

\begin{problem}
Докажите, что функция:
 $f(x)= 1 - \dfrac{x^2}{2} + \dfrac{x^3}{3} - (1+x)\cdot e^{-x},$ \; строго положительна при $x > 0;$

 $f(x) = \dfrac{2}{2x+1}-\ln\left(1 + \dfrac{1}{x}\right),$ \; строго отрицательна при $x > 0;$

$f(x) = \ln\left(1 + \dfrac{1}{x}\right) - \dfrac{1}{\sqrt{x^2 + x}},$ \; строго отрицательна при $x > 0.$

\end{problem}

\begin{problem}
Пусть $f : (-1;1) \, \rightarrow \, \mathbb{R}$ - функция, имеющая производную на $(-1;1)$ и $f^{\prime}(0) > 0$. Существует ли окрестность точки $0$, в которой функция $f$ возрастает? Существует ли такая окрестность, если функция $f^{\prime}$ непрерывна в точке $0$?
\end{problem}

\begin{problem}
Докажите, что если $\varphi$ - монотонно возрастающая дифференцируемая функция и $|f^{\prime}(x)| \leqslant |\varphi^{\prime}(x)|$ при $x \geqslant x_0$, то $|f(x) - f(x_0)| \leqslant \varphi(x) - \varphi(x_0)$ при $x \geqslant x_0$;

 \emph{1.} $\varphi(x), \psi(x)$ $n$ раз дифференцируемы при $x > x_0$;

\emph{2.} $\varphi^{(k)}(x_0)=\psi^{(k)}(x_0) \quad (k = \overline{0, n-1})$;

\emph{3.} $\varphi^{(n)}(x) > \psi^{(n)}(x)$ при $x > x_0$,

\emph{то $\varphi(x) > \psi(x)$ при $x > x_0$}.
\end{problem}

\begin{problem}
Докажите следующие неравенства:
\vspace{2ex}

\emph{(а)} $e^x>1+x,$ при $x\neq 0;$ \hspace{15ex}
\emph{(б)} \! \!\!$\left(\!1+\dfrac{1}{x}\!\right)^x\!\!\!<\!e\!<\!\!\left(\!1+\dfrac{1}{x}\!\right)^{x+1}\!\!\!\!,$ при $x>0;$

 \vspace{2ex}

\emph{(в)} $x^\alpha-1>\alpha(x-1)$, при $\alpha \geqslant 2, x>1;$   \hspace{.5ex}
\emph{(г)} $1+2\ln{x}\leqslant x^2,$ при $x>0;$

 \vspace{2ex}

\emph{(д)}  $x^{\sqrt{x+1}}>(x + 1)^{\sqrt{x}}$, при $x\geqslant 9;$  \hspace{6.4ex}
\emph{(е)} $\dfrac{\tg{x_2}}{\tg{x_1}} > \dfrac{x_2}{x_1}$, при $0 < x_1 < x_2 < \dfrac{\pi}{2}.$
\end{problem}

\begin{problem}
  Докажите неравенство: 
$$\pi x (1 - x) < \sin{\pi x} \le 4x (1 - x), \;\; 0 < x < 1.$$
  Сравните $\tg{\big(\sin{x}\big)}$ и $\sin{\big(\tg{x}\big)}$ для всех $x \in \left(0; \dfrac{\pi}{2}\right).$
\vspace{-1ex}
\end{problem}

\begin{problem}
Докажите, что для любого $x > 0$ и любого $n \in \mathbb{N}$ справедливо неравенство:
$$
0 < e^x - \sum_{k = 0}^n \dfrac{x^k}{k!} < \dfrac{x}{n} (e^x - 1).
$$

%\begin{wrapfigure}{r}{235pt}
%\vspace{2ex}
%\hspace{-2ex}
%\includegraphics[width=0.55\textwidth]{ris_26aa.jpg}
%\end{wrapfigure}
\end{problem}

\begin{problem}
 Найдите интервалы выпуклости и точки перегиба функции:
\vspace{3ex}

\emph{(а)}  $f(x)=\dfrac{x^2}{(x-1)^3};$ \hspace{25ex}
\emph{(б)} $f(x)=\dfrac{|x-1|}{x\sqrt{x}};$
\vspace{2ex}

\emph{(в)} $y=f(x), \,\,\, \begin{cases}
x=1+\ctg{t}\\
y=\dfrac{\cos{2t}}{\sin{t}}
\end{cases}\!\!\!\!\!\!,
$ $0<t<\pi.$
\end{problem}

\begin{problem}
Покажите, что кривая $y = \dfrac{x+1}{x^2+1}$ имеет \emph{три} точки перегиба, лежащие на одной прямой.
\end{problem}

\begin{problem}
 Пусть $f(x)$ дважды дифференцируема на луче $[a , +\infty)$, причём:
$$\textbf{1)} \; f(a) = A > 0; \qquad \textbf{2)} \; f^{\prime}(a) < 0; \qquad \textbf{3)} \; f^{\prime \prime}(x) \le 0, \; \text{при} \;\; x > a.$$
Докажите, что уравнение $f(x)=0$ имеет единственный корень на луче $(a, +\infty)$.
\end{problem}

\begin{problem}
 Пусть для дважды дифференцируемой функции $f(x)$ выполнено
$\lim\limits_{x\rightarrow x_0+0} f(x) = 0, \lim\limits_{x\rightarrow +\infty} f(x) = 0$.
Докажите, что на луче $(x_0, +\infty)$ имеется по меньшей мере одна точка $\xi,$ что $f^{\prime \prime}(\xi)=0$.
\end{problem}

\begin{problem}
При каком выборе параметра $h$ "кривая вероятности" \newline $y=\dfrac{h}{\sqrt{\pi}}e^{-h^2x^2} (h>0)$ имеет точки перегиба $x=\pm \sigma$?
\end{problem}

\begin{problem}
Покажите, что
\begin{itemize}
\item[-] функции  $x^n \,\, (n>1), \;\, e^x, \;\, x \ln{x}$ выпуклы вниз на $(0,+\infty)$,
\item[-] функции $x^n \,\, (0<n<1), \;\, \ln{x}$ выпуклы вверх на $(0, +\infty)$.
\end{itemize}
\end{problem}

\vspace{-3ex}
\begin{problem}[Неравенство Йенсена]
\index{Неравенство!Йенсена}

Пусть $f$ - выпуклая вниз на интервале $(a; b)$ функция.
Для $\forall  n \in \mathbb{N},  n  \ge 2$, $\; \forall \{x_1, x_2, \ldots ,x_n\}~\in~(a; b)$ и
$\;\, \forall \{\alpha_1, \alpha_2, \ldots ,\alpha_n\} \in [0; 1], \; \sum\limits_{k=1}^n \alpha_k = 1.$ Докажите, что выполнено следующее неравенство:
\vspace{-3ex}

$$
f\left(\sum\limits_{k=1}^n \alpha_k x_k\right) \le \sum\limits_{k=1}^n \alpha_k f(x_k).
$$
\end{problem}
\vspace{-3ex}

\begin{problem}
 Дана возрастающая функция $f : [0; 1] \mapsto [0; 1].$ Докажите, что найдётся такая точка $x \in [0; 1],$ что $f(x) = x.$
\end{problem}

\vspace{2ex}

\begin{remark}
 Данную задачу полезно сравнить с номером \textbf{9.16}.
\end{remark}

\vspace{-1ex}
\begin{problem}
 Пусть $f$ -- действительнозначная функция. Обозначим через $Im[f] - $ множество значений данной функции. Докажите или опровергните следующие утверждения:

Если $f(x) - $ непрерывна и $Im[f] = \mathbb{R},$ тогда $f(x) - $ монотонна;

Если $f(x) - $ монотонна и $Im[f] = \mathbb{R},$ тогда $f(x) - $ непрерывна;

Если $f(x) - $ монотонна и непрерывна, тогда $Im[f] = \mathbb{R}$.
\end{problem}

\vspace{-1ex}
\begin{problem}
Пусть функция $f$ дважды непрерывно дифференцируема на отрезке $[0; 1].$ Докажите, что функцию $f$ можно представить в виде разности двух выпуклых вниз функций.
\end{problem}

\begin{problem}
 Пусть $f : [0; 1] \mapsto \mathbb{R} - $ непрерывная функция. Докажите, что найдётся подмножество $\textbf{X} \subset [0; 1]$ мощности континуум, на котором функция $f$ монотонна.
\end{problem}

\vspace{-1ex}

\begin{problem}

 Существует ли монотонная функция $f : [0; 1] \mapsto [0; 1]$ такая, что для каждого $y \in [0; 1]$ уравнение $f(x) = y$ имеет несчётное множество решений $x$.
непрерывно дифференцируемая функция $f : [0; 1] \mapsto [0; 1]$ такая, что для каждого $y \in [0; 1]$ уравнение $f(x) = y$ имеет несчётное множество решений $x$?
\end{problem}

\begin{problem}
Докажите, что ограниченная и выпуклая на отрезке $[a; b]$ функция всюду на нём непрерывна и имеет производные слева и справа.
\end{problem}

\end{document}
