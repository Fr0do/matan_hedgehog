\documentclass[10pt]{article}
\pdfoutput=1
\usepackage{Listo4ki}


\begin{document}

\setcounter{footnote}{0}

\setcounter{section}{0}

\setcounter{part}{8}
	%\maketitle
	\flushbottom
	\newpage
	\pagestyle{fancynotes}
	\part[Предел функции. Условие его существования]{Предел функции. Условие его существования}
	\begin{margintable}\vspace{.8in}\footnotesize
		\begin{tabularx}{\marginparwidth}{|X}
		Секция~\ref{sec:theory_9}. Теория\\
		Секция~\ref{sec:problems_9}. Задачи\\
		%Section~\ref{sec:license}. Margins\\
		\end{tabularx}
	\end{margintable}
	\section[Теория]{Теория}\label{sec:theory_9}

\begin{definition}
Если каждому значению переменной $x\in\textbf{X}$ ставится в соответствие по известному закону $f$ некоторое (единственное) число $y\in \textbf{Y}$, то говорят, что  \textbf{на множестве $\textbf{X}$ задана функция $y = f(x)$}. В этом случае множество $\textbf{X}$ называется \textbf{областью определения}, а множество $\textbf{\{y\}}$ - \textbf{областью значений} данной функции.
\end{definition}

\begin{definition}
$f(x)$ - \textbf{ограничена на множестве} \textbf{X}, если 
$$\exists M>0 : \forall x\in \textbf{X} \,\,\, |f(x)|<M.$$ (\textbf{ограничена сверху и снизу})
\end{definition}

\begin{definition} Введём определения \textbf{точных верхней} и \textbf{нижней граней}

\medskip

$M^* \!= \sup\limits_\textbf{X}\!f(x)$, если 
\emph{1)} $\forall x \in \textbf{X} \, f(x) \leqslant M^*$; \, \emph{2)} $\forall \varepsilon > 0 \; \exists x' \in \textbf{X}: f(x') > M^* - \varepsilon$.

\medskip

$M_*=\inf\limits_\textbf{X}\! f(x)$, если 
\emph{1)}  $\forall x \in \textbf{X} \, f(x)\geqslant M_*$;
\, \emph{2)} $\forall \varepsilon > 0 \; \exists x''\in \textbf{X} : f(x'') < M_* + \varepsilon$.
\end{definition}

\begin{definition}
$f$ - \textbf{монотонно возрастает} $\Leftrightarrow \forall x_1, x_2\!\in\!\textbf{X}: x_1\!<\!x_2 \Rightarrow f(x_1)\!<\!f(x_2)$;

\medskip

$f$ - \textbf{монотонно не убывает} $\Leftrightarrow \forall x_1, x_2 \in \textbf{X}: x_1<x_2 \Rightarrow f(x_1)\leqslant f(x_2)$;
\medskip

$f$ - \textbf{монотонно убывает} $\Leftrightarrow \forall x_1, x_2 \in \textbf{X}: x_1<x_2 \Rightarrow f(x_1)> f(x_2)$;

\medskip

$f$ - \textbf{монотонно не возрастает} $\Leftrightarrow \forall x_1, x_2 \in \textbf{X}: x_1<x_2 \Rightarrow f(x_1)\geqslant f(x_2)$.
\end{definition}

\begin{definition} $b=\lim\limits_{x\rightarrow a} f(x)$ (\textbf{по Коши}), если
$$\forall \varepsilon > 0 \;\; \exists \, \delta(\varepsilon)>0 : \, \forall x \in \textbf{X}, \;\; 0<|x-a|<\delta \, \Rightarrow \, |f(x) - b| < \varepsilon.$$
\index{Предел функции!по Коши}
\end{definition}

\begin{definition} $b=\lim\limits_{x\rightarrow a} f(x)$ (\textbf{по Гейне}), если
$$\forall \{x_n\}, \; x_n \in \textbf{X}, \; x_n \neq a, \;\; x_n \rightarrow a \;\; f(x_n) \rightarrow b.$$
\index{Предел функции!по Гейне}
\end{definition}

\begin{proposition}
$b=\lim\limits_{x\rightarrow a} f(x)$ (Коши)  $\Longleftrightarrow$ $b=\lim\limits_{x\rightarrow a} f(x)$ (Гейне).
\end{proposition}

\begin{definition}
$\lim\limits_{x\rightarrow a} f(x)=+\infty \!\Leftrightarrow\! \forall E>0 \exists \delta(E): \forall x\!\in\!\textbf{X}, 0\!<\!|x-a|\!<\!\delta \Rightarrow f(x)\!>\!E$;
\medskip

$\lim\limits_{x\rightarrow \infty} f(x)=b \; \Leftrightarrow \; \forall \varepsilon > 0 \; \exists \delta(\varepsilon) \, : \; \forall x \in \textbf{X} \,\,\, |x|>\delta \; \Rightarrow \; |f(x)-b|<\varepsilon.$
\medskip

$\nexists\lim\limits_{x\rightarrow a} f(x)=b \; \Leftrightarrow \; \exists \varepsilon > 0 \; \forall \delta > 0 \; \exists x \in \textbf{X} \,\,\, 0<|x-a|<\delta \Rightarrow |f(x)-b|\geqslant \varepsilon.$
\end{definition}

\begin{definition}
Интервал, содержащий точку $x\in \mathbb{R}$, называется \textbf{окрестностью} этой точки.
\index{Окрестность точки}
\end{definition}

\begin{definition}
Точка $a$ называется \textbf{предельной точкой множества $\textbf{X}$}, если для $\, \forall \delta > 0 \,$ в каждой её окрестности $\textbf{B}_\delta(a)=(a-\delta,\,a+\delta)$ содержатся отличные от $a$ значения $x\in\textbf{X}.$
\index{Предельная точка множества}
\end{definition}

\begin{definition}
Точка множества, не являющаяся предельной, называется \textbf{изолированной точкой множества}.
\index{Изолированная точка множества}
\end{definition}

\section[Задачи]{Задачи.}\label{sec:problems_9}

\begin{problem}
\emph{\textbf{a)}} Докажите, что $x$ является предельной точкой множества $\textbf{X}~\!\subset~\!\mathbb{R}$, тогда и только тогда, когда из этой последовательности можно выделить подпоследовательность, сходящуюся к $x$;

\medskip

\emph{\textbf{б)}} Докажите, что в любой окрестности предельной точки $x$ множества $\textbf{X}$ находится не менее, чем счётное число точек множества, отличных от $x$.
\end{problem}
\begin{solution}
\emph{\textbf{a)}} Пусть
\marginnote{\footnotesize{{$\Longrightarrow$}}}
$\forall \mathbf{U}(x)$ $\exists n \in \mathbb{N}$ : $x_n \in \stackrel{\circ}{U}\!\!(x)$. Рассмотрим совокупность окрестностей $\left\{\stackrel{\circ}{U}_{1/n}\!\!(x)\right\}$. В первой окрестности $\stackrel{\circ}{U}_{1}\!\!(x)$ выберем элемент $x_{n_1}$. В  $\stackrel{\circ}{U}_{1/2}\!\!(x)$ -- элемент $x_{n_2}$, так, чтобы выполнялось $n_2 > n_1$. В $\stackrel{\circ}{U}_{1/3}\!\!(x)$ -- $x_{n_3}$, $n_3 > n_2$ и т.д.\footnote{Этот процесс можно продолжать бесконечно, т.к. в любой окрестности найдётся бесконечно много элементов последовательности $x_n$.}. В результате мы получим подпоследовательность $x_{n_k}$, %последовательности $x_{n}$,
которая сходится к $x$, т.к.~$|x_{n_k}-x|<\frac1k$, $\forall k \in \mathbb{N}$ и $\frac1k \xrightarrow[k \to \infty]{} 0$.

\medskip

Пусть
\marginnote{\footnotesize{{$\Longleftarrow$}}}
$\exists x_{n_k} \xrightarrow[k \to \infty]{}x$. Тогда по определению предела числовой последовательности в каждой окрестности $\mathbf{U}(x)$ лежит бесконечно много элементов подпоследовательности $x_{n_k}$ (все, начиная с некоторого номера). Остаётся заметить, что каждый элемент подпоследовательности является элементом последовательности $x_{n}$.

\medskip

\emph{\textbf{б)}} \texttt{От противного}. Пусть этих точек -- конечное число. Тогда уменьшая окрестность точки $x$ конечное число раз, можно получить окрестность \texttt{без} точек данного множества. \texttt{Противоречие с определением предельной точки}.
\end{solution}



\end{document}
